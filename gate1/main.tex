\let\negmedspace\undefined
\let\negthickspace\undefined
\documentclass[journal]{IEEEtran}
\usepackage[a5paper, margin=10mm, onecolumn]{geometry}
%\usepackage{lmodern} % Ensure lmodern is loaded for pdflatex
\usepackage{tfrupee} % Include tfrupee package

\setlength{\headheight}{1cm} % Set the height of the header box
\setlength{\headsep}{0mm}     % Set the distance between the header box and the top of the text

\usepackage{gvv-book}
\usepackage{gvv}
\usepackage{cite}
\usepackage{amsmath,amssymb,amsfonts,amsthm}
\usepackage{algorithmic}
\usepackage{graphicx}
\usepackage{textcomp}
\usepackage{xcolor}
\usepackage{txfonts}
\usepackage{listings}
\usepackage{enumitem}
\usepackage{mathtools}
\usepackage{gensymb}
\usepackage{comment}
\usepackage[breaklinks=true]{hyperref}
\usepackage{tkz-euclide} 
\usepackage{listings}

% \usepackage{gvv}                                        
\def\inputGnumericTable{}                                 
\usepackage[latin1]{inputenc}                                
\usepackage{color}                                            
\usepackage{array}                                            
\usepackage{longtable}                                       
\usepackage{calc}  
\usepackage{circuitikz}
\usepackage{multirow}                                         
\usepackage{hhline}                                           
\usepackage{ifthen}                                           
\usepackage{lscape}
\usepackage{tikz}
\begin{document}

\bibliographystyle{IEEEtran}
\vspace{3cm}


\renewcommand{\thefigure}{\theenumi}
\renewcommand{\thetable}{\theenumi}
\setlength{\intextsep}{10pt} % Space between text and floats


\numberwithin{equation}{enumi}
\numberwithin{figure}{enumi}
\renewcommand{\thetable}{\theenumi}

\title{Gate PH-2007}
\author{AI24BTECH11034 Tanush Sri Sai Petla
}
\maketitle
\renewcommand{\thefigure}{\theenumi}
\renewcommand{\thetable}{\theenumi}
\begin{enumerate}[start=69]
    \item In the circuit shown, the voltage at test point P is 12 V and the voltage between gate and source is -2 V. The value of $R$ (in k$\Omega$) is

\begin{figure}[!ht]
\centering
\resizebox{0.5\textwidth}{!}{%
\begin{circuitikz}
\tikzstyle{every node}=[font=\normalsize]
\draw (10.25,13.25) to[R] (13.25,13.25);
\draw (9.75,11) to[R] (13.25,11);
\draw (13.25,11) to[short] (13.25,13.25);
\draw (8.75,13.25) to[short] (9.75,13.25);
\draw (9.75,13.25) to[short] (9.75,12.75);
\draw (10.25,13.25) to[short] (10.25,12.75);
\draw (9.5,12.75) to[short] (10.25,12.75);
\draw [->, >=Stealth] (10,11) -- (10,12.75);
\draw (8.75,13.25) to[R] (7,13.25);
\draw (9.75,11) to[R] (7,11);
\draw (7,13.25) to[short] (7,11);
\draw (9.5,12.75) to[short] (10.5,12.75);
\draw (7,13.25) to[short, -o] (7,14.25) ;
\draw (8.75,13.25) to[short, -o] (8.75,14.25) ;
\draw  (10,13) ellipse (0.5cm and 0.75cm);
\draw (13.25,11) to (13.25,10.75) node[ground]{};
\node [font=\normalsize] at (9.5,14.5) {P};
\node [font=\normalsize] at (7.75,12.75) {2 k$\Omega$};
\node [font=\normalsize] at (8.5,10.5) {R};
\node [font=\normalsize] at (11.75,13.75) {4 k$\Omega$};
\node [font=\normalsize] at (11.75,10.5) {42 k$\Omega$};
\node [font=\normalsize] at (6.25,14.5) {$V_{DD}$=16V};
\end{circuitikz}
}%
\label{fig:my_label}
\end{figure}

    \begin{multicols}{4}
        \begin{enumerate}
            \item 42
            \item 48
            \item 56
            \item 70
        \end{enumerate}
    \end{multicols}

    \item When an input voltage $V_i$, of the form shown, is applied to the circuit given below, the output voltage $V_o$ is of the form:

\begin{figure}[h]
\centering
\resizebox{0.5\textwidth}{!}{%   % Adjusted width here
\begin{circuitikz}
\tikzstyle{every node}=[font=\normalsize]
\draw (1.75,9.25) to[R] (4.5,9.25);
\draw (4.5,9.25) to[short] (6.25,9.25);
\draw (1.75,7.5) to[short] (6.5,7.5);
\draw (5.5,8) to[battery1] (5.5,7.5);
\draw (5.5,9.25) to[D] (5.5,8);
\draw [<->, >=Stealth, dashed] (6.5,9.25) -- (6.5,7.5);
\draw [<->, >=Stealth, dashed] (1.75,9.25) -- (1.75,7.5);
\node [font=\normalsize] at (6.75,8.5) {$V_0$};
\node [font=\normalsize] at (2,8.5) {$V_1$};
\node [font=\normalsize] at (3.25,8.75) {R};
\node [font=\normalsize] at (4.75,8) {3V};
\end{circuitikz}
}%
\label{fig:my_label}
\end{figure}
\begin{figure}[!ht]
\centering
\resizebox{0.5\textwidth}{!}{%
\begin{circuitikz}
\tikzstyle{every node}=[font=\large]
\draw [->, >=Stealth] (4.5,11.75) -- (14,11.75);
\draw (5,11.75) to[short] (6.75,13.5);
\draw (6.75,13.5) to[short] (10.25,10);
\draw (10.25,10) to[short] (12,11.75);
\node [font=\large] at (3.5,11.75) {0V};
\draw [dashed] (6.75,13.5) -- (5,13.5);
\draw [dashed] (10.25,10) -- (5.5,10);
\node [font=\large] at (3.25,13.5) {12V};
\node [font=\large] at (3.25,10) {-12V};
\end{circuitikz}
}%
\label{fig:my_label}
\end{figure}

    \begin{enumerate}
        \item \begin{figure}[!ht]
\resizebox{0.3\textwidth}{!}{%
\begin{circuitikz}
\tikzstyle{every node}=[font=\normalsize]
\foreach \x in {0,...,0}{
  \draw  (6.5+\x*1,10.25) -- ++(0.5,1) -- ++ (0.5, -1);
}
\foreach \x in {0,...,0}{
  \draw  (8.5+\x*1,10.25) -- ++(0.5,1) -- ++ (0.5, -1);
}
\draw [->, >=Stealth] (6.25,10.25) -- (10,10.25);
\node [font=\normalsize] at (5.75,10.25) {0V};
\draw [dashed] (6.5,11.25) -- (7.5,11.25);
\node [font=\normalsize] at (5.75,11.25) {12V};
\end{circuitikz}
}%
\label{fig:my_label}
\end{figure}
        
        \item \begin{figure}[ht]
\resizebox{0.3\textwidth}{!}{%
\begin{circuitikz}
\tikzstyle{every node}=[font=\normalsize]
\draw (7.25,12.5) to[short] (8,12.5);
\foreach \x in {0,...,0}{
  \draw  (8+\x*1,12.5) -- ++(0.5,1) -- ++ (0.5, -1);
}
\draw (9,12.5) to[short] (9.75,12.5);
\foreach \x in {0,...,0}{
  \draw  (9.75+\x*1,12.5) -- ++(0.5,1) -- ++ (0.5, -1);
}
\draw (10.75,12.5) to[short] (11.25,12.5);
\draw [->, >=Stealth] (7.25,11.75) -- (12,11.75);
\node [font=\normalsize] at (6.25,11.75) {0V};
\node [font=\normalsize] at (6.25,12.5) {3V};
\node [font=\normalsize] at (6.25,13.5) {12V};
\draw [dashed] (8.5,13.5) -- (7.5,13.5);
\end{circuitikz}
}%
\label{fig:my_label}
\end{figure} 

       \item \begin{figure}[H]

\resizebox{0.3\textwidth}{!}{%
\begin{circuitikz}
\tikzstyle{every node}=[font=\normalsize]
\draw (7.25,12.5) to[short] (8,12.5);
\foreach \x in {0,...,0}{
  \draw  (9.75+\x*1,12.5) -- ++(0.5,1) -- ++ (0.5, -1);
}
\draw (10.75,12.5) to[short] (11.25,12.5);
\node [font=\normalsize] at (6.25,11.75) {0V};
\node [font=\normalsize] at (6.25,13.5) {12V};
\draw [dashed] (8.5,13.5) -- (7.5,13.5);
\draw [->, >=Stealth] (7.25,12) -- (11.75,12);
\foreach \x in {0,...,0}{
  \draw  (8+\x*1,12.5) -- ++(0.5,1) -- ++ (0.5, -1);
}
\draw (9,12.5) to[short] (9.75,12.5);
\node [font=\normalsize] at (6.25,12.5) {2.3V};
\end{circuitikz}
}%
\label{fig:my_label}
\end{figure}
 
        
        \item \begin{figure}[H]

\resizebox{0.3\textwidth}{!}{%
\begin{circuitikz}
\tikzstyle{every node}=[font=\normalsize]
\draw [->, >=Stealth] (7.75,12.25) -- (13.25,12.25);
\draw [dashed] (9,11.25) -- (8,11.25);
\node [font=\normalsize] at (6.75,12.25) {0V};
\node [font=\normalsize] at (6.75,11.25) {-12V};
\draw (8.25,12.25) to[short] (9.25,11.25);
\draw (9.25,11.25) to[short] (10.25,12.25);
\draw (11.25,12.25) to[short] (12.25,11.25);
\draw (12.25,11.25) to[short] (13.25,12.25);
\end{circuitikz}
}%
\label{fig:my_label}
\end{figure}


    \end{enumerate}



  $$\textbf{Common Data Questions }$$  
\textbf{Common Data for Questions 71, 72, 73:} \\
A particle of mass $m$ is confined in the ground state of a one-dimensional box, extending from $x = -2L$ to $x = +2L$. The wavefunction of the particle in this state is 
\begin{align*}
    \psi(x) = \psi_0 \cos\brak{\frac{\pi x}{4L}}
\end{align*}
where $\psi_0$ is a constant.

    \item The normalization factor $\psi_0$ of this wavefunction is 
    \begin{multicols}{4}
        \begin{enumerate}
            \item $\frac{\sqrt{2}}{L}$
            \item $\frac{1}{4L}$
            \item $\frac{1}{\sqrt{2L}}$
            \item $\frac{1}{L}$
        \end{enumerate}
    \end{multicols}

\item The energy eigenvalue corresponding to this state is 
\begin{multicols}{4}
 \begin{enumerate}
    \item $\frac{\hbar^2 \pi^2}{2mL^2}$
    \item $\frac{\hbar^2 \pi^2}{4mL^2}$
    \item $\frac{\hbar^2 \pi^2}{16mL^2}$
    \item $\frac{\hbar^2 \pi^2}{32mL^2}$
\end{enumerate}
\end{multicols}

\item The expectation value of $p^2$ ($p$ is the momentum operator) in this state is 
\begin{multicols}{4}
\begin{enumerate}
    \item 0
    \item $\frac{\hbar^2 \pi^2}{32L^2}$
    \item $\frac{\hbar^2 \pi^2}{16L^2}$
    \item $\frac{\hbar^2 \pi^2}{8L^2}$
\end{enumerate}
\end{multicols}

\textbf{Common Data for Questions 74, 75:} \\
The Fresnel relations between the amplitudes of incident and reflected electromagnetic waves at an interface between air and a dielectric of refractive index $\mu_r$ are:
\begin{align*}
    \frac{E_\parallel^\text{reflected}}{E_\parallel^\text{incident}} = \frac{\mu_r \cos r - \cos i}{\mu_r \cos r + \cos i} 
    \quad \text{and} \quad
    \frac{E_\perp^\text{reflected}}{E_\perp^\text{incident}} = \frac{\cos i - \mu_r \cos r}{\cos i + \mu_r \cos r}
\end{align*}
The subscripts $\parallel$ and $\perp$ refer to polarization, parallel and normal to the plane of incidence respectively. Here, $i$ and $r$ are the angles of incidence and refraction respectively.

\item The condition for the reflected ray to be completely polarized is
\begin{multicols}{4}
\begin{enumerate}
    \item $\mu \cos i = \cos r$
    \item $\cos i = \mu \cos r$
    \item $\mu \cos i = -\cos r$
    \item $\cos i = -\mu \cos r$
\end{enumerate}
\end{multicols}

\item For normal incidence at an air-glass interface with $\mu = 1.5$, the fraction of energy reflected is given by
\begin{multicols}{4}
\begin{enumerate}
    \item 0.40
    \item 0.20
    \item 0.16
    \item 0.04
\end{enumerate}
\end{multicols}
$$\textbf{Linked Answer Questions: Q.76 to Q.85 carry two marks each.}$$

\textbf{Statement for Linked Answer Questions 76 and 77:}\\
In the laboratory frame, a particle P of rest mass $m_e$ is moving in the positive x direction with a speed of $\frac{5c}{19}$. It approaches an identical particle Q, moving in the negative x direction with a speed of $\frac{2c}{5}$.


\item The speed of the particle P in the rest frame of the particle Q is
\begin{multicols}{4}
\begin{enumerate}
\item $\frac{7c}{95}$
\item $\frac{13c}{85}$
\item $\frac{3c}{5}$
\item $\frac{63c}{95}$
\end{enumerate}
\end{multicols}


\item The energy of the particle P in the rest frame of the particle Q is
\begin{multicols}{4}
\begin{enumerate}
\item $\frac{1}{2}m_{g}c^{2}$
\item $\frac{5}{4}m_{0}c^{2}$
\item $\frac{19}{13}m_{0}c^{2}$
\item $\frac{11}{2}m$
\end{enumerate}
\end{multicols}
    \textbf{Statement for Linked Answer Questions 78 and 79}

The atomic density of a solid is $5.85\times10^{32}m^{-3}$. Its electrical resistivity is $1.6\times10^{-2}\Omega~m.$

Assume that electrical conduction is described by the Drude model (classical theory), and that each atom contributes one conduction electron.


\item The drift mobility (in $m^{2}V^{-1}s.^{-1}$) of the conduction electrons is
\begin{multicols}{4}
\begin{enumerate}
\item $6.67\times10^{-3}$
\item $6.67\times10^{-6}$
\item $7.63\times10^{-1}$
\item $7.63\times10^{-9}$
\end{enumerate}
\end{multicols}

\item The relaxation time (mean free time), in seconds, of the conduction electrons is
\begin{multicols}{4}
\begin{enumerate}
\item $3.98\times10^{-13}$
\item $3.79\times10^{-14}$
\item $2.84\times10^{-12}$
\item $2.64\times10^{-11}$
\end{enumerate}
\end{multicols}
    \textbf{Statement for Linked Answer Questions 80 and 81:}

A sphere of radius R carries a polarization $\overrightarrow{P}=kr$, where k is a constant and $\overrightarrow{r}$ is measured from the centre of the sphere.

\item The bound surface and volume charge densities are given, respectively, by
\begin{multicols}{2}
\begin{enumerate}
\item $-k\abs{\overrightarrow{r}}$ and $3k$
\item $k\abs{\overrightarrow{r}}$ and $-3k$
\item $k\abs{\overrightarrow{r}}$ and $4\pi kR$
\item $-k\abs{\overrightarrow{r}}$ and $-4\pi kR$
\end{enumerate}
\end{multicols}

\item The electric field $\overrightarrow{E}$ at a point outside the sphere is given by
\begin{multicols}{4}
\begin{enumerate}
\item $\overrightarrow{E}=0$
\item $\overrightarrow{E}=\frac{kR\brak{R^2-r^2}}{4\pi\epsilon_0r^3}\hat{r}$
\item $\overrightarrow{E}=\frac{kR\brak{R^2-r^2}}{4\pi\epsilon_0r^3}\hat{r}$
\item $\overrightarrow{E}=\frac{3k\brak{r-R}}{4\pi\epsilon_0r^4}\hat{r}$
\end{enumerate}
\end{multicols}

    \textbf{Statement for Linked Answer Questions 82 and 83:}

An ensemble of quantum harmonic oscillators is kept at a finite temperature $T=\frac{1}{k_B\beta}$.

\item The partition function of a single oscillator with energy levels $\brak{n+\frac{1}{2}}\hbar\omega$ is given by
\begin{multicols}{4}
\begin{enumerate}
\item $Z=\frac{e^{-\beta\hbar\omega/2}}{1-e^{-\beta\hbar\omega}}$
\item $Z=\frac{e^{-\beta\hbar\omega}}{1-e^{-\beta\hbar\omega}}$
\item $Z=\frac{1}{1-e^{-\beta\hbar\omega}}$
\item $Z=\frac{1}{1-e^{-2\beta\hbar\omega-\mu}}$
\end{enumerate}
\end{multicols}

\item The average number of energy quanta of the oscillators is given by
\begin{multicols}{4}
\begin{enumerate}
\item $\langle n \rangle = \frac{1}{e^{\beta\hbar\omega}-1}$
\item $\langle n \rangle = \frac{e^{-\beta\hbar\omega}}{e^{\beta\hbar\omega}-1}$
\item $\langle n \rangle = \frac{e^{-2\beta\hbar\omega}}{e^{2\beta\hbar\omega}+1}$
\item $\langle n \rangle = \frac{1}{e^{\beta\hbar\omega}+1}$
\end{enumerate}
\end{multicols}

 \textbf{Statement for Linked Answer Questions 84 and 85:}

A 16 $\mu A$ beam of alpha particles, having cross-sectional area $10^{-4}m^{2}$, is incident on a rhodium target of thickness 1 $\mu m$. This produces neutrons through the reaction $^{103}Rh + \alpha \rightarrow ^{106}Pd + 3n$.

\item The number of alpha particles hitting the target per second is
\begin{multicols}{4}
\begin{enumerate}
\item $0.5\times10^{14}$
\item $1.0\times10^{14}$
\item $2.0\times10^{20}$
\item $4.0\times10^{19}$
\end{enumerate}
\end{multicols}

\item The neutrons are observed at the rate of $1.806\times10^{8}s^{-1}$. If the density of rhodium is approximated as $10^{4}kg~m^{-3}$, the cross-section for the reaction (in barns) is
\begin{multicols}{4}
\begin{enumerate}
\item 0.1
\item 0.2
\item 0.4
\item 0.8
\end{enumerate}
\end{multicols}




\end{enumerate}
\end{document}