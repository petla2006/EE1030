%iffalse
\let\negmedspace\undefined
\let\negthickspace\undefined
\documentclass[journal,12pt,twocolumn]{IEEEtran}
\usepackage{cite}
\usepackage{amsmath,amssymb,amsfonts,amsthm}
\usepackage{algorithmic}
\usepackage{graphicx}
\usepackage{textcomp}
\usepackage{xcolor}
\usepackage{txfonts}
\usepackage{listings}
\usepackage{enumitem}
\usepackage{mathtools}
\usepackage{gensymb}
\usepackage{comment}
\usepackage[breaklinks=true]{hyperref}
\usepackage{tkz-euclide} 
\usepackage{listings}
\usepackage{gvv}   
\usepackage{multicol}
\def\inputGnumericTable{}                                 
\usepackage[latin1]{inputenc}                                
\usepackage{color}                                            
\usepackage{array}                                            
\usepackage{longtable}                                       
\usepackage{calc}                                             
\usepackage{multirow}                                         
\usepackage{hhline}                                           
\usepackage{ifthen}                                           
\usepackage{lscape}

\newtheorem{theorem}{Theorem}[section]
\newtheorem{problem}{Problem}
\newtheorem{proposition}{Proposition}[section]
\newtheorem{lemma}{Lemma}[section]
\newtheorem{corollary}[theorem]{Corollary}
\newtheorem{example}{Example}[section]
\newtheorem{definition}[problem]{Definition}
\newcommand{\BEQA}{\begin{eqnarray}}
\newcommand{\EEQA}{\end{eqnarray}}
\newcommand{\define}{\stackrel{\triangle}{=}}
\theoremstyle{remark}
\newtheorem{rem}{Remark}
\begin{document}

\bibliographystyle{IEEEtran}
\vspace{3cm}

\title{Complex Numbers}
\author{AI24BTECH11034 - Tanush Sri Sai Petla$^{*}$% <-this % stops a space
}
\maketitle{Section-A}

\maketitle{Fill in the blanks}\\
1. If the expression    \\       
$\frac{\sbrak{\sin{\brak{{\frac{x}{2}}}}+\cos{\brak{{\frac{x}{2}}}}+i\tan{\brak{x}}}}{1+2i\sin{\brak{\frac{x}{2}}}}$\hfill\brak{1987 - 2 Marks}\\
is real,then the set of all possible values of $x$ is..... \\
2.  For any two complex numbers $z_1,z_2$ and any real number $a$ and $b$.

$\abs{az_1-bz_2}^2+\abs{bz_1+az_2}^2$ =.....  \hfill\brak{1988 - 2 Marks}\\
3.  If $a$,$b$,$c$ are the numbers between 0 and 1 such that the points $z_1=a+i$, $z_2=1+bi$ and $z_3=0$ form an equilateral triangle,then $a=....$ and $b=.....$ \hfill\brak{1989 - 2 Marks}\\
4.  $ABCD$ is a rhombus. Its diagonals $AC$ and $BD$ intersect at the point M and satisfy $BD=2AC$. If the points D and M represent the complex numbers $1+i$ and $2-i$ respectively, then A represents the complex number..... or..... \hfill\brak{1993 - 2 Marks}\\
5.  Suppose $Z_1$,$Z_2$,$Z_3$ are the vertices of an equilateral triangle inscribed in the circle $\abs{z}=2$.If $Z_1=1+i\sqrt{3}$ then $Z_2=....$ ,$Z_3=.... $ \hfill\brak{1994 - 2 Marks}\\
\maketitle{B   True/False}\\
1.  For complex number $z_1=x_1+iy_1 $ and $z_2=x_2+iy_2$, we write $z_1\cap z_2$, if $x_1\leq x_2 $and $y_1\leq y_2 $ then for all complex numbers $z$ with $1\cap z$ , we have $\frac{1-z}{1+z} \cap 0$  \hfill\brak{1981 - 2 Marks}\\
2.  If the complex numbers $z_1$, $z_2$ and $z_3 $ represent the vertices of an equilateral triangle such that $ \abs{z_1} = \abs{z_2}=\abs{z_3} $ then $z_1+z_2+z_3 = 0$ \hfill}brak{1984 - 1 Mark}\\
3.  If three complex numbers are in A.P. then they lie on a circle on the complex plane. \hfill\brak{1985 - 1 Mark}\\
4.  The cube roots of unity when represented on Argand diagram form the vertices of an equilateral triangle. \hfill\brak{1988 - 1 Mark}\\
\maketitle{C  MCQs with One Correct Answer}\\
  1. If the cube roots of unity are 1,$\omega $, $\omega^2$ , then the roots of the equation $\brak{x+1}^8 = 0$ are                     \hfill\brak{1979}
\begin{multicols}{2}
\item $(a)$ $-1$, $i+2\omega$,$1+2\omega^2$
\item $(b)$ $-1$ ,$1-2\omega$ ,$1-2\omega^2$
\item $(c)$ $ -1 , -1 ,-1 $ 
\item $(d)$ None of these
\end{multicols}
 2. The smallest positive integer for which
    $\brak{\frac{1+i}{1-i}}^n = 1$ is  \hfill\brak{1980}   
\begin{multicols}{2}
\item $(a)$ $n=8$
\item $(b)$ $n=16$
\item $(c)$ $n=12$
\item $(d)$ None of these
\end{multicols}
 3. The complex number $z= x+iy$ which satisfy the equation \hfill\brak{1981 - 2 Marks}
     $\abs{\frac{z-5i}{z+5i}} = 1 $lie on 
\begin{multicols}{2}
\item $(a)$ the x-axis
\item $(b)$ the straight line $y=5$
\item $(c)$ a circle passing through the origin 
\item $(d)$ None of these
 \end{multicols}
 4. If $z=(\frac{\sqrt{3}}{2} + \frac{i}{2})^5$ + $\brak{\frac{\sqrt{3}}{2} - \frac{i}{2}}^5$ , then \hfill{\brak{1982 - 2 Marks}
\begin{multicols}{2}
\item $(a) Re(z)=0$
\item $(b) Im(z)=0$
\item $(c) Re(z)>0, Im(z)>0$
\item $(d) Re(z)>0, Im(z)<0$
\end{multicols}
 5. The inequality $\abs{z-4}<\abs{z-2}$ represents the region given by \hfill\brak{1982 - 2 Marks}
\begin{multicols}{1}
\item $(a) Re(z)\ge0$
\item $(b) Re(z)<0$ 
\item $(c) Re(z)>0$ 
\item $(d) None of these
\end{multicols}

\end{document}
