\let\negmedspace\undefined
\let\negthickspace\undefined
\documentclass[journal]{IEEEtran}
\usepackage[a5paper, margin=10mm, onecolumn]{geometry}
%\usepackage{lmodern} % Ensure lmodern is loaded for pdflatex
\usepackage{tfrupee} % Include tfrupee package

\setlength{\headheight}{1cm} % Set the height of the header box
\setlength{\headsep}{0mm}     % Set the distance between the header box and the top of the text

\usepackage{gvv-book}
\usepackage{gvv}
\usepackage{cite}
\usepackage{amsmath,amssymb,amsfonts,amsthm}
\usepackage{algorithmic}
\usepackage{graphicx}
\usepackage{textcomp}
\usepackage{xcolor}
\usepackage{txfonts}
\usepackage{listings}
\usepackage{enumitem}
\usepackage{mathtools}
\usepackage{gensymb}
\usepackage{comment}
\usepackage[breaklinks=true]{hyperref}
\usepackage{tkz-euclide} 
\usepackage{listings}

% \usepackage{gvv}                                        
\def\inputGnumericTable{}                                 
\usepackage[latin1]{inputenc}                                
\usepackage{color}                                            
\usepackage{array}                                            
\usepackage{longtable}                                       
\usepackage{calc}  
\usepackage{circuitikz}
\usepackage{multirow}                                         
\usepackage{hhline}                                           
\usepackage{ifthen}                                           
\usepackage{lscape}
\usepackage{tikz}
\begin{document}

\bibliographystyle{IEEEtran}
\vspace{3cm}


\renewcommand{\thefigure}{\theenumi}
\renewcommand{\thetable}{\theenumi}
\setlength{\intextsep}{10pt} % Space between text and floats


\numberwithin{equation}{enumi}
\numberwithin{figure}{enumi}
\renewcommand{\thetable}{\theenumi}

\title{Gate AE-2010}
\author{AI24BTECH11034 Tanush Sri Sai Petla
}
\maketitle
\renewcommand{\thefigure}{\theenumi}
\renewcommand{\thetable}{\theenumi}
\begin{enumerate}[start=1]

\item  Isentropic efficiency $\eta_e$ of a subsonic diffuser is defined as:

(Note: 'a' represents the ambient, '2' represents the exit of the diffuser, and 's' represents an isentropic process.)

\begin{multicols}{4}
\begin{enumerate}
    \item $\frac{T_{O2s}-T_{a}}{T_{O2}-T_{a}}$
    \item $\frac{T_{O2s}+T_{a}}{T_{O2}+T_{a}}$
    \item $\frac{P_{O2s}-P_{a}}{P_{o2}-P_{a_s}}$
    \item $\frac{P_{a}-P_{O2s}}{P_{a}-P_{O2}}$
\end{enumerate}
\end{multicols}

\item Two position vectors are indicated by $\overline{V}_1 = \cbrak{\begin{matrix} x_1 \\ y_1 \end{matrix}}$ and $\overline{V}_2 = \cbrak{\begin{matrix} x_2 \\ y_2 \end{matrix}}$. If $a^2 + b^2 = 1$, then the operation $\overline{V}_2 = \begin{bmatrix} a & -b \\ b & a \end{bmatrix} \overline{V}_1$
   amounts to obtaining the position vector $\overline{V}_2$ from $\overline{V}_1$ by

\begin{enumerate}
    \item translation
    \item rotation
    \item magnification
    \item combination of translation, rotation, and magnification
\end{enumerate}

\item An aircraft is climbing at a constant speed in a straight line at a steep angle of climb. The load factor it sustains during the climb is:

\begin{multicols}{2}
\begin{enumerate}
    \item equal to $1.0$
    \item greater than $1.0$
    \item positive but less than $1.0$
    \item dependent on the weight of the aircraft
\end{enumerate}
\end{multicols}

\item In a general case of a homogeneous material under thermo-mechanical loading, the number of distinct components of the state of stress is:
\begin{multicols}{4}
\begin{enumerate}
    \item $3$ 
    \item $4$ 
    \item $5$ 
    \item $6$
\end{enumerate}
\end{multicols}

\item The linear second-order partial differential equation:
$5 \frac{\partial^2 \phi}{\partial x^2} + 3 \frac{\partial^2 \phi}{\partial x \partial y} + 2 \frac{\partial^2 \phi}{\partial y^2} + 9 = 0$ is:
\begin{multicols}{2}
\begin{enumerate}
    \item Parabolic
    \item Hyperbolic
    \item Elliptic
    \item None of the above
\end{enumerate}
\end{multicols}

\item All other factors remaining constant, if the weight of an aircraft increases by 30\%, then the takeoff distance increases by approximately:
\begin{multicols}{4}
\begin{enumerate}
    \item 15\%
    \item 30\%
    \item 70\% 
    \item 105\%
\end{enumerate}
\end{multicols}

\item A vertical slender rod is suspended by a hinge at the top and hangs freely. It is heated until it attains a uniform temperature. Neglecting the effect of gravity, the rod has
\begin{multicols}{2}
\begin{enumerate}
    \item Stress but no strain
    \item Strain but no stress
    \item Both stress and strain
    \item Neither stress nor strain
\end{enumerate}
\end{multicols}

 \item An aircraft stalls at a speed of 40 m/s in straight and level flight. The slowest speed at which this aircraft can execute a level turn at a bank angle of 60 degrees is:
    \begin{multicols}{4}
    \begin{enumerate}
    \item  28.3 m/s
    \item  40.0 m/s
    \item  56.6 m/s
    \item  80.0 m/s
\end{enumerate}
 \end{multicols}

\item The eigen-values of a real symmetric matrix are always
    \begin{multicols}{2}
        \begin{enumerate}
        \item positive
        \item imaginary
        \item real
        \item complex conjugate pairs
        \end{enumerate}
        \end{multicols}
        
         
\item The concentrations of a certain chemical species at time $t$ in a chemical reaction is described by the differential equation $\frac{dx}{dt}+kx=0$, with $x\brak{t=0}=x_{0}$. Given that $e$ is the base of the natural logarithms, the concentration $x$ at $t=\frac{1}{k}$
       \begin{multicols}{2}
        \begin{enumerate}
        \item falls to the value $0.5x_{0}$
        \item rises to the value $2x_{0}$
        \item falls to the value $\frac{x_{0}}{e}$
        \item rises to the value $ex_{0}$
        \end{enumerate}
        \end{multicols}

\item The definite integral $\int_{-1}^{+1}\frac{dx}{x^{2}}$
\begin{multicols}{4}
    \begin{enumerate}
    \item does not exist
    \item is equal to 2
    \item is equal to 0
    \item is equal to -2
    \end{enumerate}
    \end{multicols}
    
\item The absolute ceiling of an aircraft is the altitude above which it:
    \begin{enumerate}
    \item can never reach
    \item cannot sustain level flight at a constant speed
    \item can perform accelerated flight as well as straight and level flight at a constant speed
    \item can perform straight and level flight at a constant speed only
    \end{enumerate}

\item A thin rectangular plate made of isotropic material which satisfies the octahedral \brak{\text{i.e., Von Mises/Distortion energy}} failure criterion has yield strength of 200 MPa under uniaxial tension. As shown in the figure, if it is loaded with uniform tension of 150 MPa along the x-direction, the maximum uniform tensile stress that can be applied along the y-direction before the plate starts yielding is about

    \begin{figure}[H]
\centering
\resizebox{0.5\textwidth}{!}{%
\begin{circuitikz}
\tikzstyle{every node}=[font=\small]
\draw  (9.75,12.5) rectangle (12.5,11.25);
\draw [->, >=Stealth] (10,12.5) -- (10,13);
\draw [->, >=Stealth] (10.25,12.5) -- (10.25,13);
\draw [->, >=Stealth] (10.5,12.5) -- (10.5,13);
\draw [->, >=Stealth] (10.75,12.5) -- (10.75,13);
\draw [->, >=Stealth] (11,12.5) -- (11,13);
\draw [->, >=Stealth] (11.25,12.5) -- (11.25,13);
\draw [->, >=Stealth] (11.5,12.5) -- (11.5,13);
\draw [->, >=Stealth] (11.75,12.5) -- (11.75,13);
\draw [->, >=Stealth] (12,12.5) -- (12,13);
\draw [->, >=Stealth] (12.25,12.5) -- (12.25,13);
\draw [->, >=Stealth] (12.5,12.25) -- (13,12.25);
\draw [->, >=Stealth] (12.5,12) -- (13,12);
\draw [->, >=Stealth] (12.5,11.75) -- (13,11.75);
\draw [->, >=Stealth] (12.5,11.5) -- (13,11.5);
\draw [->, >=Stealth] (12.25,11.25) -- (12.25,10.75);
\draw [->, >=Stealth] (12,11.25) -- (12,10.75);
\draw [->, >=Stealth] (11.75,11.25) -- (11.75,10.75);
\draw [->, >=Stealth] (11.5,11.25) -- (11.5,10.75);
\draw [->, >=Stealth] (11.25,11.25) -- (11.25,10.75);
\draw [->, >=Stealth] (11,11.25) -- (11,10.75);
\draw [->, >=Stealth] (10.75,11.25) -- (10.75,10.75);
\draw [->, >=Stealth] (10.5,11.25) -- (10.5,10.75);
\draw [->, >=Stealth] (10.25,11.25) -- (10.25,10.75);
\draw [->, >=Stealth] (10,11.25) -- (10,10.75);
\draw [->, >=Stealth] (9.75,12.25) -- (9.25,12.25);
\draw [->, >=Stealth] (9.75,12) -- (9.25,12);
\draw [->, >=Stealth] (9.75,11.75) -- (9.25,11.75);
\draw [->, >=Stealth] (9.75,11.5) -- (9.25,11.5);
\draw [->, >=Stealth, dashed] (10.75,11.75) -- (11.75,11.75);
\draw [->, >=Stealth, dashed] (10.75,11.75) -- (10.75,12.5);
\node [font=\small] at (10.5,12.25) {y};
\node [font=\small] at (11.25,11.5) {x};
\node [font=\small] at (11.25,13.5) {$\sigma_{xy}$};
\node [font=\small] at (14.25,12) {$\sigma_{M}=150 MPa $};
\end{circuitikz}
}%
\label{fig:my_label}
\end{figure}           
        \begin{multicols}{4}
        \begin{enumerate}
            \item $227$ MPa
            \item $77$ MPa
            \item $87$ MPa
            \item $114$ MPa
        \end{enumerate}
        \end{multicols}


\end{enumerate}
\end{document}