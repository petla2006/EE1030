\let\negmedspace\undefined
\let\negthickspace\undefined
\documentclass[journal]{IEEEtran}
\usepackage[a5paper, margin=10mm, onecolumn]{geometry}
%\usepackage{lmodern} % Ensure lmodern is loaded for pdflatex
\usepackage{tfrupee} % Include tfrupee package

\setlength{\headheight}{1cm} % Set the height of the header box
\setlength{\headsep}{0mm}     % Set the distance between the header box and the top of the text

\usepackage{gvv-book}
\usepackage{gvv}
\usepackage{cite}
\usepackage{amsmath,amssymb,amsfonts,amsthm}
\usepackage{algorithmic}
\usepackage{graphicx}
\usepackage{textcomp}
\usepackage{xcolor}
\usepackage{txfonts}
\usepackage{listings}
\usepackage{enumitem}
\usepackage{mathtools}
\usepackage{gensymb}
\usepackage{comment}
\usepackage[breaklinks=true]{hyperref}
\usepackage{tkz-euclide} 
\usepackage{listings}

% \usepackage{gvv}                                        
\def\inputGnumericTable{}                                 
\usepackage[latin1]{inputenc}                                
\usepackage{color}                                            
\usepackage{array}                                            
\usepackage{longtable}                                       
\usepackage{calc}  
\usepackage{circuitikz}
\usepackage{multirow}                                         
\usepackage{hhline}                                           
\usepackage{ifthen}                                           
\usepackage{lscape}
\usepackage{tikz}
\begin{document}

\bibliographystyle{IEEEtran}
\vspace{3cm}


\renewcommand{\thefigure}{\theenumi}
\renewcommand{\thetable}{\theenumi}
\setlength{\intextsep}{10pt} % Space between text and floats


\numberwithin{equation}{enumi}
\numberwithin{figure}{enumi}
\renewcommand{\thetable}{\theenumi}

\title{Gate PH-2010}
\author{AI24BTECH11034 Tanush Sri Sai Petla
}
\maketitle
\renewcommand{\thefigure}{\theenumi}
\renewcommand{\thetable}{\theenumi}
\begin{enumerate}
\item The solution of the differential equation for $y\brak{t}$:  $\frac{d^2y}{dt^2} - y = 2 \cosh\brak{t}$, subject to the initial conditions $y\brak{0} = 0$ and $\frac{dy}{dt}|_{t=0} = 0$, is:
\begin{multicols}{2}
\begin{enumerate}
    \item $\frac{1}{2}\cosh\brak{t} + t\sinh\brak{t}$
    \item $-\sinh\brak{t} + t\cosh\brak{t}$
    \item $t\cosh\brak{t}$
    \item $t\sinh\brak{t}$
\end{enumerate}
\end{multicols}

\item Given the recurrence relation for the Legendre polynomials:
\begin{align*}
\brak{2n+1}P_n\brak{x} = \brak{n+1}P_{n+1}\brak{x} + nP_{n-1}\brak{x}
\end{align*}
which of the following integrals has a non-zero value?
\begin{multicols}{2}
\begin{enumerate}
    \item $\int_{-1}^{1} x^2 P_n\brak{x} P_{n+1}\brak{x} dx$
    \item $\int_{-1}^{1} x P_n\brak{x} P_{n+2}\brak{x} dx$
    \item $\int_{-1}^{1} x \sbrak{P_n\brak{x}}^2 dx$
    \item $\int_{-1}^{1} x^2 P_n\brak{x} P_{n+2}\brak{x} dx$
\end{enumerate}
\end{multicols}

\item For a two-dimensional free electron gas, the electronic density n and the Fermi energy $E_f$ are related by:
\begin{multicols}{2}
\begin{enumerate}
\item n = $\frac{\brak{2mE_f}^{\frac{3}{2}}}{3\pi^2\hbar^3}$\\

\item n = $\frac{mE_f}{\pi\hbar^2}$

\item n = $\frac{mE_f}{2\pi\hbar^2}$\\

\item n = $\frac{2^{\frac{3}{2}}\brak{mE_f}^{\frac{1}{2}}}{\pi\hbar}$

\end{enumerate}
\end{multicols}

\item Far away from any of the resonance frequencies of a medium, the real part of the dielectric permittivity is
\begin{enumerate}
\item Always independent of frequency
\item Monotonically decreasing with frequency
\item Monotonically increasing with frequency
\item A non-monotonic function of frequency
\end{enumerate}

\item The ground state wavefunction of a deuteron is in a superposition of s and d states. Which of the following is NOT true as a consequence?
\begin{enumerate}
\item It has a non-zero quadrupole moment
\item The neutron-proton potential is non-central
\item The orbital wavefunction is not spherically symmetric
\item The Hamiltonian does not conserve the total angular momentum
\end{enumerate}

\item The first three energy levels of $^{228}\text{Th}_{90}$ are shown below:

\begin{figure}[H]
\centering
\resizebox{0.4\textwidth}{!}{%
\begin{circuitikz}
\tikzstyle{every node}=[font=\small]
\draw (8.25,12.25) to[short] (11,12.25);
\draw (8.25,11.75) to[short] (11,11.75);
\draw (8.25,11.25) to[short] (11,11.25);
\node [font=\small] at (11.5,11.25) {0 keV};
\node [font=\small] at (11.75,11.75) {57.5 keV};
\node [font=\small] at (11.75,12.25) {187 keV};
\node [font=\small] at (7.75,11.25) {$0^+$};
\node [font=\small] at (7.75,11.75) {$2^+$};
\node [font=\small] at (7.75,12.25) {$4^+$};
\end{circuitikz}
}%
\label{fig:my_label}
\end{figure}
The expected spin-parity and energy of the next level are given by:
\begin{multicols}{2}
\begin{enumerate}
\item $\brak{6^+, 400 \text{ keV}}$
\item $\brak{6^+, 300 \text{ keV}}$
\item $\brak{2^+, 400 \text{ keV}}$
\item $\brak{4^+, 300 \text{ keV}}$
\end{enumerate} 
\end{multicols}

\item The quark content of $\Sigma^+, K, \pi$ and p is indicated:
\begin{align*}
|\Sigma^+\rangle &= |uus\rangle; \quad |K^+\rangle = |us\rangle; \quad |\pi^+\rangle = |ud\rangle; \quad |p\rangle = |uud\rangle.
\end{align*}
In the process, $\pi^{-} + p \rightarrow K^{-} + \Sigma^{\prime}$, considering strong interactions only, which of the following statements is true?

\begin{enumerate}
    \item The process is allowed because $\Delta S = 0$.
    \item The process is allowed because $\Delta I_{y} = 0$.
    \item The process is not allowed because $\Delta S \neq 0$ and $\Delta I_{z} \neq 0$.
    \item The process is not allowed because the baryon number is violated.
\end{enumerate}

\item The three principal moments of inertia of a methanol (CH$_3$OH) molecule have the property $I_x = I_y = I$ and $I_z \neq I$. The rotational energy eigenvalues are
\begin{multicols}{2}
\begin{enumerate}
    \item $\frac{\hbar^2}{2I}l\brak{l+1} + \frac{\hbar^2 m_i^2}{2}\brak{\frac{1}{I_i} - \frac{1}{I}}$\\
    \item $\frac{\hbar^2}{2I}l\brak{l+1}$
    \item $\frac{\hbar^2 m_i^2}{2}\brak{\frac{1}{I_i} - \frac{1}{I}}$\\
    \item $\frac{\hbar^2}{2I}l\brak{l+1} + \frac{\hbar^2 m_i^2}{2}\brak{\frac{1}{I_i} + \frac{1}{I}}$
\end{enumerate}
\end{multicols}
\item A particle of mass $m$ is confined in the potential 

\begin{equation*}
V(x) = \begin{cases}
\frac{1}{2}m\omega^2 x^2 & \text{for } x > 0, \\
\infty & \text{for } x \leq 0.
\end{cases}
\end{equation*}

Let the wavefunction of the particle is given by

\begin{align*}
\psi(x) = -\frac{1}{\sqrt{5}}\psi_0 + \frac{2}{\sqrt{5}}\psi_1,
\end{align*}

where $\psi_0$ and $\psi_1$ are the eigenfunctions of the ground state and the first excited state, respectively. The expectation value of the energy is
\begin{figure}[H]
\centering
\resizebox{0.5\textwidth}{!}{%
\begin{circuitikz}
\tikzstyle{every node}=[font=\normalsize]
\draw [->, >=Stealth] (9,10.75) -- (12.75,10.75);
\draw [->, >=Stealth] (9,10.75) -- (9,13.5);
\begin{scope}[rotate around={-135:(11.5,13)}]
\draw[domain=11.5:14.75,samples=100,smooth] plot (\x,{1*sin(1*\x r -11.5 r ) +13});
\end{scope}
\node [font=\normalsize] at (8.75,10.5) {0};
\node [font=\normalsize] at (12.25,10.25) {x};
\node [font=\normalsize] at (8,13.25) {$v\brak{x}$};
\end{circuitikz}
}%
\label{fig:my_label}
\end{figure}

\begin{multicols}{4}
\begin{enumerate}
\item $\frac{31}{10}\hbar\omega$ 
\item $\frac{25}{10}\hbar\omega$ 
\item $\frac{13}{10}\hbar\omega$
\item $\frac{11}{10}\hbar\omega$
\end{enumerate}
\end{multicols}

\item Match the typical spectra of stable molecules with the corresponding wave-number range:
\begin{multicols}{2}
			\begin{enumerate}[label=(\Alph*)]
                
				\item Electronic spectra
				\item Rotational spectra
                    \item Molecular dissociation
			\end{enumerate}
			\columnbreak
			\begin{enumerate}[label=(\arabic*)]
				\item $10^6 cm^{-1}$ and above
				\item $10^5 - 10^6 cm^{-1}$
				\item $10^0 - 10^2 cm^{-1}$
                    
			\end{enumerate}
		\end{multicols}

      \begin{multicols}{2}
        \begin{enumerate}
            \item $A-2,B-1,C-3$
            \item $A-2,B-3,C-1$
            \item $A-3,B-2,C-1$
            \item $A-1,B-2,C-3$
        \end{enumerate}
    \end{multicols}
 
\item Consider the operations $P: \vec{r} \rightarrow -\vec{r}$ (parity) and $T: t \rightarrow -t$ (time-reversal). For the electric and magnetic fields $\vec{E}$ and $\vec{B}$, which of the following set of transformations is correct?
\begin{multicols}{2}
\begin{enumerate}
\item $P: \vec{E} \to -\vec{E}, \vec{B} \to \vec{B}$;\\ $T: \vec{E} \to \vec{E}, \vec{B} \to -\vec{B}$
\item $P: \vec{E} \to \vec{E}, \vec{B} \to \vec{B}$;\\ $T: \vec{E} \to \vec{E}, \vec{B} \to \vec{B}$
\item $P: \vec{E} \to -\vec{E}, \vec{B} \to \vec{B}$;\\ $T: \vec{E} \to -\vec{E}, \vec{B} \to -\vec{B}$
\item $P: \vec{E} \to \vec{E}, \vec{B} \to -\vec{B}$; \\$T: \vec{E} \to -\vec{E}, \vec{B} \to \vec{B}$
\end{enumerate}
\end{multicols}

\item Two magnetic dipoles of magnitude $m$ each are placed in a plane as shown.
\begin{figure}[H]
\centering
\resizebox{0.3\textwidth}{!}{%
\begin{circuitikz}
\tikzstyle{every node}=[font=\normalsize]
\draw [dashed] (8.25,13.75) -- (8.25,10);
\draw (8.25,10) to[short] (12.75,14.5);
\draw [dashed] (12.75,14.5) -- (8.25,14.5);
\draw [dashed] (8.25,14.5) -- (8.25,13.75);
\draw [->, >=Stealth] (12.25,14.5) -- (13.25,14.5);
\draw [->, >=Stealth] (8.25,9.75) -- (8.25,10.5);
\node [font=\normalsize] at (11.5,12) {d};
\node [font=\normalsize] at (7.5,10) {m};
\node [font=\normalsize] at (12.75,15) {m};
\node [font=\normalsize] at (12.75,14) {2};
\node [font=\normalsize] at (8.5,10) {1};
\node [font=\normalsize] at (11.5,14) {$45^\circ$};
\node [font=\normalsize] at (8.75,11.25) {$45^\circ$};
\draw (11.75,14.5) to[short] (12.25,14);
\draw (8.25,11) to[short] (8.75,10.5);
\end{circuitikz}
}%
\label{fig:my_label}
\end{figure}

The energy of interaction is given by:
\begin{multicols}{2}
\begin{enumerate}
\item Zero
\item $\dfrac{\mu_0}{4\pi} \dfrac{m^2}{d^3}$
\item $\dfrac{3\mu_0}{2\pi} \dfrac{m^2}{d^3}$
\item $-\dfrac{3\mu_0}{8\pi} \dfrac{m^2}{d^3}$
\end{enumerate}
\end{multicols}

\item Consider a conducting loop of radius $a$ and total loop resistance $R$ placed in a region with a magnetic field $B$, thereby enclosing a flux $\Phi_0$. The loop is connected to an electronic circuit as shown, the capacitor being initially uncharged.
\begin{figure}[H]
    \centering
    \resizebox{0.4\textwidth}{!}{%
    \begin{circuitikz}
        \tikzstyle{every node}=[font=\normalsize]
        \draw  (7.75,13) circle (1cm);
        \draw (8.75,13.25) to[short] (12.5,13.25);
        \draw (8.75,13) to[short] (12.5,13);
        \draw (12.5,13.25) to[short] (12.5,13.75);
        \draw (12.5,13) to[short] (12.5,12.5);
        \draw [short] (16.25,13.25) -- (15,14.25);
        \draw [short] (16.25,13.25) -- (15,12);
        \draw [short] (15,14.25) -- (15,12);
        \draw (12.5,13.75) to[short] (15,13.75);
        \draw (12.5,12.5) to[short] (15,12.5);
        \node at (14.25,13.75) [circ] {};
        \draw (14.25,13.75) to[short] (14.25,15);
        \draw (14.25,15) to[C] (17,15);
        \node at (14.25,12.5) [circ] {};
        \draw (14.25,12.5) to (14.25,11.75) node[ground]{};
        \draw (16.25,13.25) to[short] (17.5,13.25);
        \draw (17,15) to[short] (17,13.25);
        \node at (17,13.25) [circ] {};
        \draw  (6,14.5) rectangle (9.5,11.25);
        \node [font=\LARGE] at (6.25,14.25) {$\times$};
        \node [font=\LARGE] at (7,14.25) {$\times$};
        \node [font=\LARGE] at (7.75,14.25) {$\times$};
        \node [font=\LARGE] at (8.5,14.25) {$\times$};
        \node [font=\LARGE] at (9.25,14.25) {$\times$};
        \node [font=\LARGE] at (6.25,13.5) {$\times$};
        \node [font=\LARGE] at (7.5,13.5) {$\times$};
        \node [font=\LARGE] at (8.5,13.5) {$\times$};
        \node [font=\LARGE] at (7,13.5) {$\times$};
        \node [font=\LARGE] at (9.25,13.5) {$\times$};
        \node [font=\LARGE] at (6.25,12.5) {$\times$};
        \node [font=\LARGE] at (7,12.5) {$\times$};
        \node [font=\LARGE] at (7.75,12.5) {$\times$};
        \node [font=\LARGE] at (8.5,12.5) {$\times$};
        \node [font=\LARGE] at (9.25,12.5) {$\times$};
        \node [font=\LARGE] at (6.25,11.75) {$\times$};
        \node [font=\LARGE] at (7,11.75) {$\times$};
        \node [font=\LARGE] at (7.75,11.75) {$\times$};
        \node [font=\LARGE] at (8.5,11.75) {$\times$};
        \node [font=\LARGE] at (9.25,11.75) {$\times$};
        \node [font=\LARGE] at (8.75,13.25) {\textbf{B}};
        \node [font=\normalsize] at (15.5,15.75) {C};
        \node [font=\normalsize] at (18,13.25) {$V_{out}$};
        \node [font=\normalsize] at (15.25,13.75) {$-$};
        \node [font=\normalsize] at (15.25,12.5) {$+$};
    \end{circuitikz}
    }%
    
    \label{fig:my_label}
\end{figure}




If the loop is pulled out of the region of the magnetic field at a constant speed $v$, the final output voltage $V_{out}$ is independent of:
\begin{multicols}{2}
\begin{enumerate}
 \item $\Phi_0$
 \item $R$
 \item $u$
 \item $C$
 \end{enumerate}
 \end{multicols}

 
 \end{enumerate}
\end{document}
